%!TEX TS−program = xetex
%!TEX encoding = UTF−8 Unicode
\documentclass[a4paper, oneside, fleqn]{scrartcl}
\usepackage{amssymb}
\usepackage{amsbsy}
\usepackage{amsmath}
\usepackage{amsthm}
\usepackage{undertilde}
\usepackage{xunicode}
\usepackage{fontspec}
\usepackage{xltxtra}
\usepackage{polyglossia}
\setmainlanguage{english}
\setromanfont[Mapping=tex-text]{Linux Libertine O}
\setsansfont[Mapping=tex-text]{Linux Biolinum O}
\setmonofont[Mapping=tex-text, Scale=0.8]{DejaVu Sans Mono}
% \setmonofont[Mapping=tex-text, Scale=0.9]{Courier New}
%\setromanfont[Mapping=tex-text]{Times New Roman}
%\setromanfont[Mapping=tex-text]{Arial}
\usepackage{units}
\usepackage{graphicx}
\usepackage{float}
\usepackage{color}
% \usepackage{transparent}
\usepackage{listings}

\renewcommand{\labelenumi}{\alph{enumi})}
\renewcommand{\labelenumii}{\roman{enumii})}
%\renewcommand{\vec}[1]{\ensuremath{\mathbf{#1}}} % for vectors
\renewcommand{\vec}[1]{\ensuremath{\boldsymbol{#1}}} % for vectors
\newcommand{\mat}[1]{\ensuremath{\utilde{#1}}} % for matrices
\newcommand{\curl}[1]{\ensuremath{\mathbf{\nabla} \times #1}} % 
\newcommand{\abs}[1]{\ensuremath{\left| #1 \right|}} % 
\newcommand{\re}[1]{\ensuremath{\text{Re}\left\{ #1 \right\}}} % 
\renewcommand{\exp}[1]{\ensuremath{\text{e}^{ #1 }}} % 
% \def\svgwidth{5cm}
% \def\unitlength{10cm}
\begin{document}
\title{Cooperation Agreement}
\subject{Homo Mobilis — Eksperter i Team}
\author{Erik Moen — Fantastic 4.}
\maketitle
\section*{Group participants}
\begin{itemize}
  \item SHAHMANSOORI, Arash
  \item NORDAL, Andreas
  \item MOEN, Erik
  \item AAKRE, Kristine Breiteig
  \item ZHANG, Haijiao
\end{itemize}

\section*{Agreements}

\subsection*{Goals}
We have agreed to target a final grade of B or better for the EiT course, which should reflect on the quality of work done by the group members.

\subsection*{Practical}
\paragraph{Workload}
Workload should be about 10 hours per week, or approximately 5 hours plus the mandatory meeting.

\paragraph{Meetings}
Mandatory participation on weekly meetings in the basement of the Elektro-building every Wednesday at 9. At these meetings every participant's progress the past week shall be presented briefly, if possible.

Everyone will be on time at planned meetings, any absence must be announced. Electronic mail is a suggested method of making this announcement.

\paragraph{Structure}
We will have ``subdeadlines'' along the way as we work.

Work should be completed three (3) days ahead of the corresponding subdeadline. For the final deadline, work should be completed two (2) weeks ahead of the deadline.

Tasks will be divided amond the group members, who will report their progress to the rest of the group at weekly meetings.

Any problems completing assigned tasks must be communicated to the group in order to compensate or give feedback. The leader will follow up the tasks.

\subsection*{Process}
\paragraph{Leader}
The group will have a leader. The leader changes every week, starting week 4. Leadership is rotated through the list of group participants sorted by given name.

The leader shall:
\begin{itemize}
  \item Facilitate cooperation, including
  \begin{itemize}
    \item Assigning tasks
    \item Include all group members in dialogue
  \end{itemize}

  \item Lead meetings
  
  Responsible for the creation of
  \begin{itemize}
    \item Agenda, presented to the group well ahead of time.
    \item Meeting log
  \end{itemize}

  \item Arrange meetings
  \item Lead decisionmaking, preferably so every member is satisfied with the decision.
\end{itemize}

Should the leader find it troubling to communicate with certain group members, Kristine will be available for assistance in the matter.

\paragraph{Writing of final report}
The group will divide into two smaller subgroups near the end of the semester. One subgroup will focus on the project report while the other subgroup will focus on the process report.

Before the reports are handed in, the groups will switch and review each other's reports.

\paragraph{Decisions}
Decisions within the group can be made by majority vote.



\end{document}
