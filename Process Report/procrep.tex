%!TEX TS−program = xetex
%!TEX encoding = UTF−8 Unicode
\documentclass[a4paper, oneside, fleqn, halfparskip]{scrartcl}
\usepackage{amssymb}
\usepackage{amsbsy}
\usepackage{amsmath}
\usepackage{amsthm}
\usepackage{undertilde}
\usepackage{xunicode}
\usepackage{fontspec}
\usepackage{xltxtra}
\usepackage{polyglossia}
\setmainlanguage{english}
\setromanfont[Mapping=tex-text]{Linux Libertine O}
\setsansfont[Mapping=tex-text]{Linux Biolinum O}
\setmonofont[Mapping=tex-text, Scale=0.8]{DejaVu Sans Mono}
% \setmonofont[Mapping=tex-text, Scale=0.9]{Courier New}
%\setromanfont[Mapping=tex-text]{Times New Roman}
%\setromanfont[Mapping=tex-text]{Arial}
\usepackage{units}
\usepackage{wasysym}
\usepackage{graphicx}
\usepackage{float}
\usepackage{color}
% \usepackage{transparent}
\usepackage{listings}

\renewcommand{\labelenumi}{\alph{enumi})}
\renewcommand{\labelenumii}{\roman{enumii})}
%\renewcommand{\vec}[1]{\ensuremath{\mathbf{#1}}} % for vectors
\renewcommand{\vec}[1]{\ensuremath{\boldsymbol{#1}}} % for vectors
\newcommand{\mat}[1]{\ensuremath{\utilde{#1}}} % for matrices
\newcommand{\curl}[1]{\ensuremath{\mathbf{\nabla} \times #1}} % 
\newcommand{\abs}[1]{\ensuremath{\left| #1 \right|}} % 
\newcommand{\re}[1]{\ensuremath{\text{Re}\left\{ #1 \right\}}} % 
\renewcommand{\exp}[1]{\ensuremath{\text{e}^{ #1 }}} % 
\newcommand{\fact}{\emph{citation needed}} % 
% \def\svgwidth{5cm}
% \def\unitlength{10cm}
\begin{document}
\title{Procrep}
\subject{Homo Mobilis — Eksperter i Team}
\author{Erik Moen — Fantastic 4.}
\maketitle
%\tableofcontents

\section{In the beginning}
% [We didn't have a group]

More focus on:
- How interaction worked.
- Intra-group work presentations. (De er pådrag!)

\section{Week one}
%[Culture Dialogue.]

Subject dialogue
- list important factors
As instructed by the facilitartors, we started mapping our areas of knowledge on large posters, in order to find our individual and collctive strong and weak areas.

- Presenting the EMG idea
During the session Mr. Arash made his initial presentation of an idea concerning diagnotisation of patients using electromyography (EMG), which appeared to fit the group particularly well in addition to providing a new and challenging field for the participants. 

While other project ideas were discussed, we all believed this EMG idea would be our actual subject. [[CHECK]]

Discussion on teamwork
As a start on the creation of the team cooperation agreement we were instructed to concider and discuss what we wanted or wanted to avoid in the cooperation, in particular factors that helped create a good cooperation environment.
- [[List of key items chosen (I think there is a large paper)]]

% After finding common ground form this discussion, key items were documented and turned into a draft agreement. The following were adressed.
% - Leader (w/ responsibilities)
% - Decision method
% - Frequencies for meetings
% - Task distribution
% - Absence
% - [[argumentation]]

% - [[Reason for team contract]]
The team cooperation agreement is particularly usefull when people of different backgrounds are to cooperate, as it can form a common base to build the rest of the cooperation on. It also enables us to have a common reference as baseline for expectations from each other.

[[Important concepts included, with theory.]]
Of particular interest, the contents of the cooperation agreement describe our choice of goals, group leader style and responsibilities, method of making decisions, work division and feedback to the rest of the group. In addition it designates a head communicator, Miss. Kristine, who will help with difficulties in communication.

According to [[Sz:2002]] the method of making decisions in a group is a trade-off between efficiency and internal acceptance of the decisions made. We came to the conclusion (by full consensus) that a simple, informed majority is sufficient for decisions within the group, while the other methods discussed included full consensus and external arbitration. 

We decided to have a leader and rotate the position according to a static, repeating roster. This rotation enabled us to give the leader resonsibility for the completion of weekly tasks like creation of group logs outside of the meetings proper.

It was decided to attempt to divide the work on the two reports in two focus teams, where each team has the responsibility for one report, while taking input from the other.

\section{The second week}
%[Project rprt lect.]
%[Library]

Rectification of C.ag'mnt.
Importance of rectification?

Continuing the process of creating the team cooperation agreement, the draft agreement was reviewed. Imprecisions and contested sections were corrected to everyones satisfaction and previously omitted paragraphs were added. 

\paragraph{The choice of subject}
Before work could start, the group members had to agree on a project subject \fact. A session was held where the members attempted to find a subject to fit as well as possible to the criteria spec'd by the EiT village. 

%- What do we do!!
One of the criteria for a subject is to include the professional field of every member and exploit the available diversity. The discussion would thus be greatly helped by every participant knowing the field of study for the others, according to [rule 1]. Thus, we all spoke a bit about what our topics of study were. This application of theory greatly helped the discussion, and opened our eyes to new angles to include all of our areas of expertise.

The discussion lead us to further support of the EMG diagnostisation project, as it was found to both examplify the village theme and be simple to incorporate features examplifying our various areas of expertise. In addition to the signal processing (Arash' specialty), it can include a distribution component to take advantage of Victor's network expertise. The aquisition of the EMG from a patient is within the field of both Erik and Kristine, and once aquired the EMG signal can also be used in usefull cybernetic applications, Andreas' specialty. 

The subject also examplifies the village theme, \emph{Homo Mobilis}, by being based on the processes in one of the building stones of human movement, the muscles. 

[[check date, might be next week]]Before we left the village that day, we made a formal decision to have EMG as our project subject, in accordance with the procedure agreed upon in the cooperation agreement.

\section{La troisième semaine}
%Plane crash
The day started with a Check-in activity, where we individually rated items after their utility for keeping the group alive after a plane crash in winter in northern Canada. Afterwards we had a discussion within the group and decided (in accordance with our coop.ag'mnt) on a group answer. 

During this activity, we found that member participation varies between our group members, and (with the help of a \emph{sociogram} made by Miss Marie) interaction is better between some group members than others and some apparent polarization where some members communicated better/more than others. This fact was decomposed into a number of causes:
\begin{itemize}
  \item Good communication between participants with similar background, in particular Erik and Andreas.
  \item Different comittment to the task. In particular Arash concidered the task to be off-topic, and thus didn't want to engage in a demanding debate over the choices.
\end{itemize}
This helped us identify a need to knit the group more tightly together. Andreas suggested we attempt to take control over the situation by collecting the group for the upcoming luncheon. This suggestion fell through, partly because the intention was not communicated with sufficient fidelity.



%Café dialogue



[Chinese new year]
Due to the exact date of this village day, we were presented with a special concideration. The start of the Chinese New Year's celebrations caused Mr. Victor to wish to be excused from the group for the afternoon. It was unanimously decided that such cultural happenings should mandate absence from the group without penalty.

%Buildup to different opinions of our project's goals. (With following steering on behalf of Andreas)
%- Polarisation between Andreas and Arash.
%- Compounded by café dialogue.

Andreas felt action was needed while reviewing the day, and posted the appended note to the group, according to [rule 4] regarding ``open cards.'' [[I understand K. has detailed this]]

- Arranged to be taken to the EMG-lab next week.
- Important to get better understanding of both the available equipment and expertise, and to get a fuller understanding of the subject that is EMG. 

\section{Closing words}

\section{Different cultural backgrounds}
In a group with individuals from different countries it is important to take into consideration that there are a lot of different cultural aspects and that it is important to learn a little bit about each other when working so close together. The only knowledge about the culture and persona of the team has so far been learned during the check- in activities on Wednesday mornings. An example of the cultural differences was discovered during a check-in activity about our favorite colors on February 9. Arash said his favorite color was green and Victor later told us that green is associated with infidelity in his country. If a wife cheats on her husband, they will say he is wearing a green hat. Arash replied that they have the same expression in Iran only they use the color red. The Norwegians in the group had no recollection of a smiliar expression used in Norway. %In accordance with Scwartz ground rule nr 2 this is a very important part of the process, since not only the technological background is important when making decisions. Knowledge about other group members could lead to a quicker decisionmaking process since it probably would cause less misunderstandings. This also made clear later in the semester.

\section{The situation}
On the February 9 we violated one of Schwartz ground rules for effective teamwork. Talk openly about potentially hard or uncomfortable issues, so called undiscussable issues. This happened during a discussion about the project. Arash drew a schematic of the project and talked mainly about his part in the project, signal processing. A couple of days later there was posted a note on its learning from Andreas that he was sick of Arash not including his part in the project. Arash replied that he was very surprised when reading the note. The next meeting was as usual on Wednesday at 9 and we were not sure how the atmosphere in the group would be due to the posts. After the mandatory activities we decided to talk openly about the problems in the group and let both Andreas and Arash explain their side of the story, which solved the conflict. Conflict in a group is not necessary a negative thing as long as we turn it into an experience we can learn something from and use it to improve the co-operation in the team. 

\section{Telephone list}
When visiting the EMG lab the team got separated into two groups. The first group, Kristine and Arash, first returned to the classroom in the basement and waited for the other three. After a while we went looking for them but could not find them. Kristine returned to the room in the basement and found the rest of the group, but now Arash was not there. To avoid this kind of confusions in the future we decided to make a telephone list so it will be easier to reach the others if needed.

\section{EMG equipment}
The team had a few difficulties with using the EMG equipment and could retrieve the signal two weeks after the plan. This was due to faulty equipment. The first time the power supply of the computer was broken and had to be handed in for repair and the second time the computer was back from repair but lacked the necessary software for signal acquisition. This led to some frustration in the group and we felt that the progress of the project was going to slow. The delay with the EMG signal resulted in a slight change in the project, this was to prevent the group from lacking work and produced a task that was a bit different in the signal acquistion part but where the signal prosessing would be almost the same.

\section{Group form}
During the meeting on February 02 the group was supposed to fill out a form concerning the dynamics in the group. It was clear that there were some disagreements aspecially regarding the communication between the members and some meant that the group lacked a clear goal. After a quick discussion the group stated that the reason the answers was different, at least regarding the goals for the project, was that the goals have changed a bit along the way since there was a change from just studying EMG to including AMG as well. This made at least the part for Erik and Kristine a bit different since the signal used no longer is a electric signal but a acoustic signal. The group also decided to continue with both projects in parallel to make sure that all group memebers have something to do at all times. And if one of the goals regarding the EMG or AMG will not be able to be finished then the group have a back up plan. Although all the group members have decided to try to finish both parts since AMG is a newer research. The largest deviation about the communication was between Arash and the rest of the group and like he presented in his group log contribution that week: "I thought the answers to them should be better!" A reason for this is perhaps that he is one of the parts in most of the discussions where the communication fails. This was later brought up at the log feed-back meeting with the student assistants and after some dicsussion around the subject the group decided to be better at explaining the views, especially regarding the project. %And that the reason for these breaks in communication is perhaps due to a different technological background and the fact that the English is slightly different in pronunciation in the different countries and the fact that everyone in our group most communicate in their second language.

\section{Project}
When performing the signal aquisition, goal 1 of the EMG part, there were four of five members on the EMG lab. Erik had experience with LabView and therefore he was chosen to supervise the signal aquisition. The others in the group watched him work for the most of the time and although this was not the most efficient work method, this is an important part of the project and it is therefore important to know most of the details regarding the different parts of the project. 
\end{document}
